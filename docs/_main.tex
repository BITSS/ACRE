\PassOptionsToPackage{unicode=true}{hyperref} % options for packages loaded elsewhere
\PassOptionsToPackage{hyphens}{url}
%
\documentclass[]{book}
\usepackage{lmodern}
\usepackage{amssymb,amsmath}
\usepackage{ifxetex,ifluatex}
\usepackage{fixltx2e} % provides \textsubscript
\ifnum 0\ifxetex 1\fi\ifluatex 1\fi=0 % if pdftex
  \usepackage[T1]{fontenc}
  \usepackage[utf8]{inputenc}
  \usepackage{textcomp} % provides euro and other symbols
\else % if luatex or xelatex
  \usepackage{unicode-math}
  \defaultfontfeatures{Ligatures=TeX,Scale=MatchLowercase}
\fi
% use upquote if available, for straight quotes in verbatim environments
\IfFileExists{upquote.sty}{\usepackage{upquote}}{}
% use microtype if available
\IfFileExists{microtype.sty}{%
\usepackage[]{microtype}
\UseMicrotypeSet[protrusion]{basicmath} % disable protrusion for tt fonts
}{}
\IfFileExists{parskip.sty}{%
\usepackage{parskip}
}{% else
\setlength{\parindent}{0pt}
\setlength{\parskip}{6pt plus 2pt minus 1pt}
}
\usepackage{hyperref}
\hypersetup{
            pdftitle={Guidelines for Computational Reproducibility in Economics},
            pdfauthor={ACRE Team},
            pdfborder={0 0 0},
            breaklinks=true}
\urlstyle{same}  % don't use monospace font for urls
\usepackage{color}
\usepackage{fancyvrb}
\newcommand{\VerbBar}{|}
\newcommand{\VERB}{\Verb[commandchars=\\\{\}]}
\DefineVerbatimEnvironment{Highlighting}{Verbatim}{commandchars=\\\{\}}
% Add ',fontsize=\small' for more characters per line
\usepackage{framed}
\definecolor{shadecolor}{RGB}{248,248,248}
\newenvironment{Shaded}{\begin{snugshade}}{\end{snugshade}}
\newcommand{\AlertTok}[1]{\textcolor[rgb]{0.94,0.16,0.16}{#1}}
\newcommand{\AnnotationTok}[1]{\textcolor[rgb]{0.56,0.35,0.01}{\textbf{\textit{#1}}}}
\newcommand{\AttributeTok}[1]{\textcolor[rgb]{0.77,0.63,0.00}{#1}}
\newcommand{\BaseNTok}[1]{\textcolor[rgb]{0.00,0.00,0.81}{#1}}
\newcommand{\BuiltInTok}[1]{#1}
\newcommand{\CharTok}[1]{\textcolor[rgb]{0.31,0.60,0.02}{#1}}
\newcommand{\CommentTok}[1]{\textcolor[rgb]{0.56,0.35,0.01}{\textit{#1}}}
\newcommand{\CommentVarTok}[1]{\textcolor[rgb]{0.56,0.35,0.01}{\textbf{\textit{#1}}}}
\newcommand{\ConstantTok}[1]{\textcolor[rgb]{0.00,0.00,0.00}{#1}}
\newcommand{\ControlFlowTok}[1]{\textcolor[rgb]{0.13,0.29,0.53}{\textbf{#1}}}
\newcommand{\DataTypeTok}[1]{\textcolor[rgb]{0.13,0.29,0.53}{#1}}
\newcommand{\DecValTok}[1]{\textcolor[rgb]{0.00,0.00,0.81}{#1}}
\newcommand{\DocumentationTok}[1]{\textcolor[rgb]{0.56,0.35,0.01}{\textbf{\textit{#1}}}}
\newcommand{\ErrorTok}[1]{\textcolor[rgb]{0.64,0.00,0.00}{\textbf{#1}}}
\newcommand{\ExtensionTok}[1]{#1}
\newcommand{\FloatTok}[1]{\textcolor[rgb]{0.00,0.00,0.81}{#1}}
\newcommand{\FunctionTok}[1]{\textcolor[rgb]{0.00,0.00,0.00}{#1}}
\newcommand{\ImportTok}[1]{#1}
\newcommand{\InformationTok}[1]{\textcolor[rgb]{0.56,0.35,0.01}{\textbf{\textit{#1}}}}
\newcommand{\KeywordTok}[1]{\textcolor[rgb]{0.13,0.29,0.53}{\textbf{#1}}}
\newcommand{\NormalTok}[1]{#1}
\newcommand{\OperatorTok}[1]{\textcolor[rgb]{0.81,0.36,0.00}{\textbf{#1}}}
\newcommand{\OtherTok}[1]{\textcolor[rgb]{0.56,0.35,0.01}{#1}}
\newcommand{\PreprocessorTok}[1]{\textcolor[rgb]{0.56,0.35,0.01}{\textit{#1}}}
\newcommand{\RegionMarkerTok}[1]{#1}
\newcommand{\SpecialCharTok}[1]{\textcolor[rgb]{0.00,0.00,0.00}{#1}}
\newcommand{\SpecialStringTok}[1]{\textcolor[rgb]{0.31,0.60,0.02}{#1}}
\newcommand{\StringTok}[1]{\textcolor[rgb]{0.31,0.60,0.02}{#1}}
\newcommand{\VariableTok}[1]{\textcolor[rgb]{0.00,0.00,0.00}{#1}}
\newcommand{\VerbatimStringTok}[1]{\textcolor[rgb]{0.31,0.60,0.02}{#1}}
\newcommand{\WarningTok}[1]{\textcolor[rgb]{0.56,0.35,0.01}{\textbf{\textit{#1}}}}
\usepackage{longtable,booktabs}
% Fix footnotes in tables (requires footnote package)
\IfFileExists{footnote.sty}{\usepackage{footnote}\makesavenoteenv{longtable}}{}
\usepackage{graphicx,grffile}
\makeatletter
\def\maxwidth{\ifdim\Gin@nat@width>\linewidth\linewidth\else\Gin@nat@width\fi}
\def\maxheight{\ifdim\Gin@nat@height>\textheight\textheight\else\Gin@nat@height\fi}
\makeatother
% Scale images if necessary, so that they will not overflow the page
% margins by default, and it is still possible to overwrite the defaults
% using explicit options in \includegraphics[width, height, ...]{}
\setkeys{Gin}{width=\maxwidth,height=\maxheight,keepaspectratio}
\setlength{\emergencystretch}{3em}  % prevent overfull lines
\providecommand{\tightlist}{%
  \setlength{\itemsep}{0pt}\setlength{\parskip}{0pt}}
\setcounter{secnumdepth}{5}
% Redefines (sub)paragraphs to behave more like sections
\ifx\paragraph\undefined\else
\let\oldparagraph\paragraph
\renewcommand{\paragraph}[1]{\oldparagraph{#1}\mbox{}}
\fi
\ifx\subparagraph\undefined\else
\let\oldsubparagraph\subparagraph
\renewcommand{\subparagraph}[1]{\oldsubparagraph{#1}\mbox{}}
\fi

% set default figure placement to htbp
\makeatletter
\def\fps@figure{htbp}
\makeatother

\usepackage{booktabs}
\usepackage[]{natbib}
\bibliographystyle{apalike}

\title{Guidelines for Computational Reproducibility in Economics}
\author{ACRE Team}
\date{2020-02-26}

\begin{document}
\maketitle

{
\setcounter{tocdepth}{1}
\tableofcontents
}
\textbf{{[}ADD ONE PAGER BETWEEN COVER AND INTRO{]}}

\hypertarget{section}{%
\chapter*{}\label{section}}
\addcontentsline{toc}{chapter}{}

\begin{verbatim}
                (1)       (2)         (3)        (4)        (5)
              scope --> assess --> improve --> robust --> extend
               ▲         |  |                   ▲
               |         |  |                   |
               |_________|  |___________________|
\end{verbatim}

\protect\hyperlink{scoping}{Scoping}

\protect\hyperlink{assessment}{Assessment}

\protect\hyperlink{improvements}{Improvement}

\protect\hyperlink{robust}{Robustness}

Extensions

\protect\hyperlink{improvements}{Outcome-level}

\protect\hyperlink{paper-level}{Paper-level}

☐ Select paper

\protect\hyperlink{describe-inputs}{☐ Describe inputs}

Output-level imp:

\protect\hyperlink{paper-level}{☐ + Version control}

\protect\hyperlink{id-analy}{☐ Analytical choices}

☐ New method

\protect\hyperlink{check-acre}{☐ Check ACRE}

\protect\hyperlink{diagram}{☐ Reproduction diagrams}

\protect\hyperlink{rd}{☐ + Raw data}

\protect\hyperlink{paper-level}{☐ + Documentation}

\protect\hyperlink{id-type}{☐ Type of choice}

☐ New data

☐ \protect\hyperlink{verify-rep-mat}{Search for rep. mat.}

\protect\hyperlink{score}{☐ Reproduction score}

\protect\hyperlink{ad}{☐ + Analysis data}

\protect\hyperlink{paper-level}{☐ + Dynamic document}

\protect\hyperlink{id-val}{☐ Choice value}

☐ New data

\protect\hyperlink{read-summ}{☐ Read and summ.}

\protect\hyperlink{ac}{☐ + Analysis code}

\protect\hyperlink{paper-level}{☐ + File structure}

\protect\hyperlink{test-rob}{☐ Justify and test alternatives}

\protect\hyperlink{outputs}{☐ Output(s)}

\protect\hyperlink{dac}{☐ Debug analysis code}

\protect\hyperlink{intensive}{☐ Depth}

\protect\hyperlink{dcc}{☐ Debug cleaning code}

Record results is Survey 1

Record results is Survey 2

Record results is Survey

\hypertarget{introduction-intro}{%
\chapter{Introduction \{: \#intro\}}\label{introduction-intro}}

Last year (2019), the \href{https://www.aeaweb.org/journals/policies/data-code/}{American Economic Association} began requiring all the materials for reproduction of any new publication. In addition to the requirements, several \href{https://aeadataeditor.github.io/aea-de-guidance/}{specific recommendations} were produced to facilitate the authors' compliance. This change in policy and guidance should radically improve the computational reproducibility of all new published research. However all the published research before 2019 did not have neither the requirements nor the guidance to develop proper reproduction materials.

As a result, several studies point towards rates of computational reproducibility in economics that range from alarmingly low \citep{galiani2018make, chang2015economics} to just low \citep{kingi2018reproducibility}. The project of Advancing Computational Reproducibility in Economics (ACRE) has the goals of improving, and having a much clearer picture of, computational reproducibility in economics.

Replication, or the process by which a study's hypotheses and findings are re-examined using different data or different methods (or both) \citep{King95} is an essential part of the scientific process that allows science to be ``self-correcting.'' Computational reproducibility, or the ability to reproduce the results, tables, and other figures using the available data, code, and materials, is a precondition for replication. Computational reproducibility is assessed through the process of reproduction. At the center of this process is the reproducer (you!), a party not involved in the production of the original paper. Reproductions sometimes involve the original author (whom we refer to as ``the author'') in cases where additional guidance and materials are needed to execute the process.

This exercise is designed for reproductions performed in Economics graduate courses or undergraduate theses, with the goal of providing common terminology and standards for conducting reproductions. The goal of reproduction, in general, is to assess and improve the computational reproducibility of published research in a way that facilitates future replication, extension, and collaboration.

\hypertarget{beyond-binary-judgments}{%
\section{Beyond binary judgments}\label{beyond-binary-judgments}}

Assessments of reproducibility can easily gravitate towards binary assessments that declare a entire paper ``reproducible'' or non-reproducible". These guidelines suggest a more nuance approach by highlighting three dimensions that make binary judgement less relevant. Firs one paper usually contains several scientific claims, and those claims can vary in computational reproducbility. Second, each claim is typically measured through several specifications. This makes easier for reproducer to hypothesize after the fact about which are the main specifications where reproducibility assessed. Finally there are several levels of reproucibility, ranging from the absence of any materials to the complete reproducibility starting from the raw data. And even for a specific claim-specification, distinguishing the appropriate level can be far more constructive than simple labeling as reproducible or not.

\begin{Shaded}
\begin{Highlighting}[]
\CommentTok{# JOEL: can you figure out why this to run int the html version of the book?}
\KeywordTok{library}\NormalTok{(DiagrammeR)}

\KeywordTok{DiagrammeR}\NormalTok{(}\StringTok{"}
\StringTok{   graph TB}
\StringTok{   Paper-->Claim1}
\StringTok{   Paper-->Claim2}
\StringTok{   Paper-->Claim3}
\StringTok{   Claim1-->Spec11}
\StringTok{   Claim1-->Spec12}
\StringTok{   Claim1-->Spec13}
\StringTok{   Claim2-->Spec21}
\StringTok{   Claim2-->Spec22}
\StringTok{   Claim2-->Spec23}
\StringTok{   Claim3-->Spec31}
\StringTok{   Claim3-->Spec32}
\StringTok{   Claim3-->Spec33}
\StringTok{   "}\NormalTok{)}
\end{Highlighting}
\end{Shaded}

I should be noticed that the highest level of reproducibility, which requires complete reproducibility starting from the raw data, is very demanding and it should not be expected of published research (especially that published before 2019). This high bar is set up as an aspiration to improve the current reproducibility of research and facilitate the transmission of knowledge in the scientific community.

\hypertarget{stages-of-the-exercise}{%
\section{Stages of the exercise}\label{stages-of-the-exercise}}

This reproduction exercise is divided into four stages, with a fifth optional stage :

\begin{enumerate}
\def\labelenumi{\arabic{enumi}.}
\item
  \textbf{Scoping}, where the reproducer defines the scope of the exercise by declaring a paper and specific output(s) on which they will focus;\\
\item
  \textbf{Assessment}, where the reproducer reviews and describes in detail the available reproduction materials (or the ``reproduction package'') and assess current levels of computational reproducibility;\\
\item
  \textbf{Improvement}, where the reproducer modifies the content and/or the organization of the reproduction materials to improve its reproducibility;\\
\item
  \textbf{Robustness checks}, where the reproducer assesses the quality of selected analytical choices from the paper; and\\
\item
  \textbf{Extension} (if applicable), where the reproducer may ``extend'' the current paper by including new methodologies or data. This step brings the reproduction exercise a step closer to replication.

\begin{verbatim}
            (1)       (2)         (3)        (4)        (5)
          scope --> assess --> improve --> robust --> extend
           ▲         |  |                   ▲
           |         |  |                   |
           |_________|  |___________________|

    Suggested level of effort:
- Graduate
  research:   5%       10%        5%         10%         70%
- Graduate
  course:    10%       25%       20%         40%         5%
- Undergrad.
  thesis:    10%       30%       40%         20%         0%
\end{verbatim}
\end{enumerate}

(\textbf{JOEL}: Please add guesstimates of level of efforts for projects that last 2 weeks, 1 month or one semester. Ask Katie for a paper form 3ie on Push button replications (one of the authors is Sayak Khatua)

Figure 1 depicts the suggested levels of effort for each stage of the exercise depending on the context in which you are performing this exercise. This process need not be chronologically linear, e.g.~you can realize that the scope of their reproduction is too ambitious and then switch to a less intensive reproduction. Later in the exercise, you can also begin testing different specifications for robustness while also assessing the current level of reproducibility.

\hypertarget{recording-the-results-of-the-exercise}{%
\section{Recording the results of the exercise}\label{recording-the-results-of-the-exercise}}

You will be asked to record the results of your reproduction as you make progress through different stages. After the \protect\hyperlink{scoping}{scoping stage}, you will be asked to complete \href{https://berkeley.qualtrics.com/jfe/form/SV_8hLHNI6LGSYchEN}{a survey} that records paper of choice and specific outputs to reproduce. In the \protect\hyperlink{assessment}{assessment stage}, you should inspect all the reproduction materials (raw data, analysis data and code), draw diagrams that connect the target output (to reproduce) with all its inputs, and score the level of reproducibility. All this information will be recorded in a \href{ADD\%20LINK}{standarized spreadsheet}. At the \href{(\#improvements)}{improvement stage}, you can record any specific improvements report any potential increase in the reported level of reproducibility in a second short survey. Finally, in the \protect\hyperlink{robust}{robustness stage} you will record any analytical choice and possible variations of it.

\hypertarget{scoping}{%
\chapter{Scoping}\label{scoping}}

In this first stage, you are assigned or choose a paper to reproduce. This section describes the different steps required to successfully define the scope of a reproduction exercise. \textbf{Use \href{https://berkeley.qualtrics.com/jfe/form/SV_8hLHNI6LGSYchEN}{Survey 1} to record your work as part of this stage.}

\hypertarget{steps}{%
\section{Steps}\label{steps}}

Before you invest any time in reading/summarizing the paper:

\hypertarget{check-acre}{%
\subsection{Check records on the ACRE platform}\label{check-acre}}

Check the \href{ADD\%20LINK}{ACRE database} for previous assessments of the same paper. If there is no previous assessment, create the first entry. If there are previous entries, check if the paper-status is ``open to reproductions'', and create a new entry\footnote{If the paper appears as ``closed to reproductions'' it means that ACRE has documented interactions between previous reproductors and original authors, showing that it is not possible to access any further required materials. }.

\hypertarget{verify-rep-mat}{%
\subsection{Verify that reproduction materials exist}\label{verify-rep-mat}}

Verify that the paper has reproduction materials (consult the AEA unofficial \href{https://social-science-data-editors.github.io/guidance/Verification_guidance.html}{Data and Code Guidance} to determine whether you have everything before you start). If yes, continue. If no, verify in our records if authors have been requested these materials before. If nobody has submitted a similar request before, contact the authors using the language suggested here, and make sure to CC (\href{mailto:acre@berkeley.edu}{\nolinkurl{acre@berkeley.edu}}). If the authors have been contacted before and the information is still missing, submit a request for closing the paper for further reproductions and choose another paper.

\hypertarget{read-sum}{%
\subsection{Read and summarize the paper}\label{read-sum}}

Depending on the duration of the exercise, it is recommended to create a short summary of the paper to reproduce, written in the reproducers own words. This serves two purposes: reminder of the key elements to focus for the reproduction, and demonstrating understanding.

Read the paper and write a two-page summary: focus on what is the main question the paper answers, its methodology and data sources.

\hypertarget{declare-scope-of-the-exercise}{%
\subsection{Declare scope of the exercise}\label{declare-scope-of-the-exercise}}

Declare if assessment will be about all or main output, and if it aims to start from raw data, or analytic data. See definitions of these concepts below.

\hypertarget{reproduction-margins}{%
\subsubsection{Reproduction margins}\label{reproduction-margins}}

\hypertarget{outputs}{%
\paragraph{Extensive margin: outputs to reproduced}\label{outputs}}

\begin{itemize}
\item
  Raw data: A data set is considered to be raw if it corresponds to the unmodified file that was obtained by the authors from the sources cited in the paper. The only possible modification that can be made to raw data, without changing its category to processed data, is that of deleting personally identifiable information.
\item
  Processed data: a raw data set that has gone through any transformation should be defined as processed data. There are two kinds of processed data:

  \begin{itemize}
  \tightlist
  \item
    \emph{Intermediate data}: a processed dataset is defined as intermediate if it is not directly used as final input for an analysis in the final paper (including appendix). Intermediate data should not contain direct identifiers\\
  \item
    \emph{Analytic data}: data will be defined as analytic if it will be used as the last input in the workflow, to produce a statistic displayed in the final paper (including the appendix).
  \end{itemize}
\end{itemize}

\hypertarget{types-of-output}{%
\subparagraph{Types of output:}\label{types-of-output}}

\begin{itemize}
\tightlist
\item
  \textbf{Tables:} Describe how to cite record location.
\item
  \textbf{Figures:} ADD\\
\item
  \textbf{In-line:} ADD
\end{itemize}

\hypertarget{intensive}{%
\paragraph{Intensive margin: depth of reproduction in terms of original data}\label{intensive}}

\begin{itemize}
\item
  \textbf{Raw data:} A data set is considered to be raw, if it corresponds to the a unmodified file that was obtained by the authors from the sources cited in the paper. The only possible modification that can be made to raw data, without changing its category to processed data, is that of deleting personally identifiable information.
\item
  \textbf{Processed data:} a raw data set that has gone through any transformation should be defined as processed data. Processed data can be separated into intermediate data and analytic data.

  \begin{itemize}
  \tightlist
  \item
    \emph{Intermediate data:} a processed dataset is defined as intermediate if it is not directly used as final input for an analysis in the final paper (including appendix). Intermediate data should not contain direct identifiers.
  \item
    \emph{Analytic data:} data will be defined as analytic if it will be used as the last input in the workflow, to produce a statistic displayed in the final paper (including the appendix).
  \end{itemize}
\end{itemize}

\hypertarget{how-to-classify-a-data-set}{%
\subparagraph{How to classify a data set?}\label{how-to-classify-a-data-set}}

To classify a data set as \textbf{raw} you can use some of the following strategies:

\begin{itemize}
\tightlist
\item
  Check if the data is stored in a folder labeled as ``raw''\\
\item
  Check if the file name has the word ``raw'' in some part of it\\
\item
  Verify that the data set is not the output of any code script in the reproduction materials\\
\item
  Verify that the same file can be independently obtained from the data source cited in the paper
\end{itemize}

To classify a data set as \textbf{analytic} you can use some of the following strategies:

\begin{itemize}
\tightlist
\item
  Check if the data is stored in a folder denominated as ``analytic'' or ``analysis''\\
\item
  Check if the file name has the word ``analytic'' or ``analysis'' in some part of it\\
\item
  Verify that the data set is the last input required to produce some of the output (formatted or unformatted) of the paper
\end{itemize}

\hypertarget{sample-strategies-for-scoping}{%
\section{Sample strategies for scoping}\label{sample-strategies-for-scoping}}

JOEL, PLEASE ADD.

\hypertarget{strategy-1}{%
\subsection{Strategy 1:}\label{strategy-1}}

\begin{itemize}
\item
  First main estimates from paper.
\item
  Then main 2 main tables and/or two main figures.
\item
  Once done with improvements stage of these outcomes, then go back to rest of the paper.
\end{itemize}

\hypertarget{strategy-2}{%
\subsection{Strategy 2:}\label{strategy-2}}

\begin{itemize}
\tightlist
\item
  get everything to run up to analytic data, then go to improvements. Once completed go back for raw data (or in parallel?)
\end{itemize}

\hypertarget{assessment}{%
\chapter{Assessment}\label{assessment}}

The goal of this stage is to provide a standardized assessment of computational reproducibility. All the details required in this section are designed to leave a record of most of the learning process behind a reproduction. Such a record facilitates incremental improvements, allowing new reproducers to pick up where others left.

First, you will be asked to provide a detailed description of all the reproduction materials. Second, you requested to connect the outputs chosen to reproduce with all of their corresponding inputs. With all this elements in place, you can then score the level of reproducibility of a specific output, and report on paper-level dimensions of reproducibility.

\textbf{Use Survey 2 to record your work as part of this step.}

\emph{Tip: }We recommend that you first focus on one specific output (e.g. ``Table 1''). After completing the assessment to this first output, you will have a much easier time translating those improvements to other outputs.

ADD: how a reproduction package \href{https://social-science-data-editors.github.io/guidance/Verification_guidance.html.}{should look like}

\hypertarget{describe-inputs}{%
\section{Describe inputs}\label{describe-inputs}}

This section explains how to list all the input materials found or referred to in the reproduction package. First, identify all data sources and connect them with raw data files (when available). Second, locate and provide a description a brief description of the analytic files. In the last step, locate, inspect and describe all the computer code used in the paper.

In addition to the concepts of \protect\hyperlink{intensive}{raw and processed data}, defined already, this section use the following concepts:

\begin{itemize}
\item
  \textbf{Cleaning code:} a script can be classified as primarily data cleaning if most of its content is dedicated to actions such as: deleting variables or observations, merging data sets, removing outliers, and reshaping the structure of the data (from long to wide or vice versa).
\item
  \textbf{Analysis code:} a script can be classified as primarily analysis code if most of its content is dedicated to actions such as: running regressions, running hypothesis tests, computing standard errors, and imputing missing values.

  \begin{itemize}
  \tightlist
  \item
    ADD DEFINITION FOR REPRODUCTION PACKAGE.
  \end{itemize}
\end{itemize}

\hypertarget{data-sources-and-raw-data}{%
\subsection{Data sources and raw data}\label{data-sources-and-raw-data}}

In the paper to reproduce, find references to all the data sources used in the analysis. A data source is usually described in a narrative form, for example the body of the paper can have text like ``\ldots{}for earnings in 2018 we use the Current Population Survey\ldots{}''-- in this example, the data source is ``Current Population Survey 2018''. It is mentioned for the first time on page 1 of the appendix, so its location should be recorded as ``A1''. Do this for all the data sources mentioned in the paper.

Next, look at the reproduction package and map the \emph{data sources} mentioned in the paper to the \emph{data files} in the available materials. In addition to looking at all the existing data files, it is recommended to review the first lines of all code files (especially cleaning code), looking for lines that call the data sets. By inspecting this scripts you might be able to understand how each different data sources is used, and possibly identify any missing files from the reproduction package.

Record all the information of this section into a \href{https://docs.google.com/spreadsheets/d/1LUIdVFH0OfR70C7z07TYeE-uWzKI_JIeWUMaYhqEKK0/edit\#gid=0\&range=A1}{assessment tool} with the following structure:

\begin{verbatim}
      Raw data information:
      |----------------------|------|-----------------------------------------------|---------------------|
      | data_source          | page | data_files                                    | known_missing       |
      |----------------------|------|-----------------------------------------------|---------------------|
      | "Current Population  | A1    | cepr_march_2018.dta                          |                     |
      | Survey 2018"         |      |                                               |                     |
      |----------------------|------|-----------------------------------------------|---------------------|
      | "DHS 2010 - 2013"    | 4    | nicaraguaDHS_2010.csv;                        | boliviaDHS_2011.csv |
      |                      |      | boliviaDHS_2010.csv; nicaraguaDHS_2011.csv;   |                     |
      |                      |      | nicaraguaDHS_2012.csv; boliviaDHS_2012.csv;   |                     |
      |                      |      | nicaraguaDHS_2013.csv; boliviaDHS_2013.csv    |                     |
      |----------------------|------|-----------------------------------------------|---------------------|
      | "2017 SAT scores"    | 4    | Not available                                 |                     |
      |----------------------|------|-----------------------------------------------|---------------------|
      | ...                  | ...  | ...                                           | ...                 |
      |----------------------|------|-----------------------------------------------|---------------------|
\end{verbatim}

\hypertarget{describe-analytic-data-sets}{%
\subsection{Describe analytic data sets}\label{describe-analytic-data-sets}}

List all the analytic files that you find in the reproduction materials and identify its location relative to the master reproduction folder\footnote{relative location takes the form \texttt{/folder\_in\_rep\_materials/sub\_folder/file.txt}, in contrast to absolute location that has the form \texttt{username/documents/projects/repros/folder\_in\_rep\_materials/sub\_folder/file.txt}}. Record all this information in a \href{https://docs.google.com/spreadsheets/d/1LUIdVFH0OfR70C7z07TYeE-uWzKI_JIeWUMaYhqEKK0/edit\#gid=1299317837\&range=A1}{assessment tool}.

On an initial review, it will probably be hard to provide a one-line description of each analytic data set. But as you progress through the exercise, return to the spreadsheet and add a one-line description of the main content in each file (for example: \texttt{all\_waves.csv} has the simple description \texttt{data\ for\ region-level\ analysis}).

The resulting report will have the following structure:

\begin{verbatim}
      Analysis data information:
      |----------------|-----------------------|--------------------------------|
      | analysis_data  | location              | description                    |
      |----------------|-----------------------|--------------------------------|
      | final_data.csv | /analysis/fig1/       | data for figure1               |
      |----------------|-----------------------|--------------------------------|
      | all_waves.csv  | /final_data/v1_april/ | data for region-level analysis |
      |----------------|-----------------------|--------------------------------|
      | ...            | ...                   | ...                            |
      |----------------|-----------------------|--------------------------------|
\end{verbatim}

\hypertarget{describe-code-scripts}{%
\subsection{Describe code scripts}\label{describe-code-scripts}}

List all the code files that you find in the reproduction materials and identify its location relative to the master reproduction folder. Review the beginning and end of each code file and identify the inputs required to successfully run the file. Example of inputs are data sets or other code scripts that are typically found at the beginning of the script (e.g.: \texttt{load}, \texttt{read}, \texttt{source}, \texttt{run}, \texttt{do} ). Examples of outputs are other data sets, or plain text files that are typically at the end of a script (e.g.: \texttt{save}, \texttt{write}, \texttt{export}). Record all this information in the \href{https://docs.google.com/spreadsheets/d/1LUIdVFH0OfR70C7z07TYeE-uWzKI_JIeWUMaYhqEKK0/edit\#gid=1617799822\&range=A1}{assessment tool}.

As you gain understanding of each code script, youy will likely find more inputs and outputs -- we encourage you update the \href{https://docs.google.com/spreadsheets/d/1LUIdVFH0OfR70C7z07TYeE-uWzKI_JIeWUMaYhqEKK0/edit\#gid=1617799822\&range=A1}{assessment tool}. Once finished with the reproduction exercise, classify each code file as analysis or cleaning type. This subjective assessment should be made base on the students' interpretation of the main role of each script.

Record all this information in a \href{https://docs.google.com/spreadsheets/d/1LUIdVFH0OfR70C7z07TYeE-uWzKI_JIeWUMaYhqEKK0/edit\#gid=1617799822\&range=A1}{assessment tool}

\begin{verbatim}
      |-------------------|------------------|---------------------|---------------------|----------------------|--------------|
      | file_name         | location         | inputs              | outputs             | description          | primary_type |
      |-------------------|------------------|---------------------|---------------------|----------------------|--------------|
      | output_table1.do  | /code/analysis/  | analysis_data01.csv | output1_part1.txt   | produces first part  | analysis     |
      |                   |                  |                     |                     | of table 1           |              |
      |                   |                  |                     |                     | (unformatted)        |              |
      |-------------------|------------------|---------------------|---------------------|----------------------|--------------|
      | data_cleaning02.R | /code/cleaninig/ | admin_01raw.csv     | analysis_data02.csv | removes outliers     | cleaning     |
      |                   |                  |                     |                     | and missing vals     |              |
      |                   |                  |                     |                     | from raw admin data  |              |
      |-------------------|------------------|---------------------|---------------------|----------------------|--------------|
      | ...               | ...              | ...                 | ...                 | ...                  | ...          |
      |-------------------|------------------|---------------------|---------------------|----------------------|--------------|
\end{verbatim}

\hypertarget{diagram}{%
\section{Connect each output to all its inputs}\label{diagram}}

Draw diagrams from output to raw data sources. For more examples of diagrams connecting final output to initial raw data, \protect\hyperlink{additional-diagrams}{see here}.

\begin{verbatim}
      table 1
        └───[code] formatting_table1.R
            ├───output1_part1.txt  
            |   └───[code] output_table1.do           
            |       └───[data] analysis_data01.csv
            |          └───[code] data_cleaning01.R
            |             └───[data] survey_01raw.csv
            └───output1_part2.txt  
                └───[code] output_table2.do           
                    └───[data] analysis_data02.csv
                       └───[code] data_cleaning02.R
                          └───[data] admin_01raw.csv  
\end{verbatim}

This diagram can be represented in data format by specifying how each component depends to its inputs. For example:

\begin{verbatim}
      Data representation of diagram behind Table 1.
      |--------|-------|-------------------|---------------------|------------|
      | ouput  | order | component         | depends_on          | inpt_type  |
      |--------|-------|-------------------|---------------------|------------|
      | table1 | 1     | table1            | formatting_table1.R | code       |
      |--------|-------|-------------------|---------------------|------------|
      | table1 | 2     |formatting_table1.R| output1_part2.txt   | output     |
      |--------|-------|-------------------|---------------------|------------|
      | table1 | 3     |formatting_table1.R| output1_part1.txt   | output     |
      |--------|-------|-------------------|---------------------|------------|
      | table1 | 4     | output_table1.do  | analysis_data01.csv | data       |
      |--------|-------|-------------------|---------------------|------------|
      | ...    | ...   | ...               | ...                 | ...        |
      |--------|-------|-------------------|---------------------|------------|
\end{verbatim}

Record all this information in the \href{https://docs.google.com/spreadsheets/d/1LUIdVFH0OfR70C7z07TYeE-uWzKI_JIeWUMaYhqEKK0/edit\#gid=1384504774\&range=A1}{assessment tool}.
In case of any difficulty translating the diagram into a spreadsheet, students can draw it with pen and paper, take a picture and upload the picture in the assessment survey.

\hypertarget{score}{%
\section{Assign a reproducibility score}\label{score}}

Once all the possible inputs have been identified, and there is a clear understanding of the connection between the outputs and inputs, it is possible to assess the output-specific level of reproducibility.

The following concepts will be used in this section:

\begin{itemize}
\item
  \textbf{Computationally Reproducible from Analytic data (CRA):} the output can be reproduced with minimal effort starting from the analytic data sets.
\item
  \textbf{Computationally Reproducible from Raw data (CRR):} the output can be reproduced with minimal effort from the raw data sets.
\item
  \textbf{Minimal effort:} spending five minutes or less in getting the code running. This five minutes do not include the computing time.
\end{itemize}

\hypertarget{levels-of-computational-reproducibility-for-a-specific-output}{%
\subsection{Levels of Computational Reproducibility for a Specific Output}\label{levels-of-computational-reproducibility-for-a-specific-output}}

We can now outline different levels of computational reproducibility. Each level is defined on the basis of data and code availability, possible improvements, and whether or not it achieves some type of reproducibility. In the next chapter we describe each possible improvement with more detail.

MAKE EXPLICIT UNDERLYING VALUATIONS:

\begin{itemize}
\tightlist
\item
  L1: The worst is to have no access to any data or code.\\
\item
  L2: Having only code is better than nothing, but worst than any other combination\\
\item
  L3: Having analytic data but not code is better than having code and no data.
\item
  L4 and L5: Conditional on having all analytic material. Reproducible is better than not
\item
  L6: Having raw and analyitic data is better than having code but no data.\\
\item
  L7: Given that there is raw and analysis data, having cleaning code only does not add much. having analysis code does add (to lvl 7 and lvl 8 if CRA)\\
\item
  L9: Having all materials that but not being able to reproduce is better than all previous cases. Achieving CRA (lvl 10) and CRR (lvl 11) are the best levels according to this criteria.
\end{itemize}

ADD ONE QUESTION AT THE END OF SURVEY TWO TO ASK REPRODUCER IF THEY AGREE WITH THIS ORDER, AND IF THEY WOULD ADD/DELETE/MODIFY ANY SPECIFIC LEVEL

The assessment is made at the output level. Hence a paper can be highly reproducible for its main results, but suffer of low reproducibility for other outputs. The assessment is on a 10-level scale, where 0 represents that, under the current circumstances, reproducers cannot access any reproduction materials and 10 represents access to all the materials and the target outcome can be reproduced starting from raw data.

\textbf{Level 1 (L1):} There are no data or code available of any type. Possible improvements include adding: raw data (+AD), analysis data (+RD), cleannig code and analysis code (+CC, +AC).

\textbf{Level 1.5 (L1.5):} There are only code scripts available, but no data of any type are available. Possible improvements include adding: raw data (+AD), analysis data (+RD).

\textbf{Level 2 (L2):} There are analytic data available, but no raw data or any type of code. Possible improvements include: adding raw data (+RD) and adding analysis code (+AC).

\textbf{Level 3 (L3):} Both analytic data sets and analysis code are available. However, the code does not run or produces different results than those of the paper (not CRA). Possible improvements include obtaining raw data (+RD) or debugging the analysis code (DAC).

\textbf{Level 4 (L4):} Both analytic data sets and analysis code are available, and they produce the same output as in the paper (yes CRA). The reproducibility package can still be improved by obtaining the original raw data sets, or by documenting the steps required to obtain those files.

\textbf{Level 5 (L5):} All data, analytic and raw, are available. However, some or all the codes for cleaning and analysis are missing. Steps for improvement include adding analysis code (+AC) and/or cleaning code (+CC).

\textbf{Level 6 (L6):} All data and analysis code are available. However, the code does not run or produces different results than those of the paper (not CRA). Possible improvements include adding the missing cleaning code (+CD) or debugging the analysis code (DAC).

\textbf{Level 7 (L7):} All data and analysis code are available, and they produce the same output as in the paper (yes CRA). The reproducibility package can still be improved by adding the missing cleaning code (+CD).

\textbf{Level 8 (L8):} All materials (raw and analysis data, and cleaning and analysis code) are available. However, the code does not run or produces different results than those of the paper (not CRR and not CRA). Possible improvements include debugging the cleaning code (DCC) or debugging the analysis code (DAC).

\textbf{Level 9 (L9):} All materials (raw and analysis data, and cleaning and analysis code) are available, and the analysis code produces the same output as in the paper (yes CRA). However the cleaning code does not run or produces different results that those of the paper (not CRR). Possible improvements include debugging the cleaning code (DCC).

\textbf{Level 10 (L10):} All materials are available and produce the same results as in the paper with minimal effort, starting from the analytic data sets (yes CRA) and from the raw data (yes CRR).

The following figure summarizes the different levels of computational reproducibility (for any given output). For each level, there will be possible improvements that have been done already (-), that can be done to move up one level of reproducibility (\texttt{✔}) or that are out of reach given the current level of reproducibility (\texttt{x}).

\begin{verbatim}
Figure 1: Levels of Computational Reproducibility and Possible Improvements   

                                           |       |               Possible improvements              |
                                           -----------------------------------------------------------
                                           | Level |+Analysis|  +Raw  |+Analysis|+Cleaning|Debug|Debug|
                                           |       |   Data  |  Data  | Code(AC)| Code(CC)|  AC |  CC |
                                           --------|---------|--------|---------|---------|-----|-----|
What data are available?                   |       |         |        |         |         |     |     |
├── None ..................................|   L1  |    ✔    |    ✔   |    ✔    |    ✔    |  x  |  x  |
├── Analytic data only. Code?              |       |         |        |         |         |     |     |
|   ├── No code or cleaning code only......|   L2  |    -    |    ✔   |    ✔    |    ✔    |  x  |  x  |
|   └── Analysis code only. Is it CRA?     |       |         |        |         |         |     |     |
|      ├── No..............................|   L3  |    -    |    ✔   |    -    |    ✔    |  ✔  |  x  |
|      └── Yes.............................|   L4  |    -    |    ✔   |    -    |    ✔    |  -  |  x  |
└── Raw & Analytic data. Code?             |       |         |        |         |         |     |     |
   ├── None ...............................|   L5  |    -    |    -   |    ✔    |    ✔    |  x  |  x  |
   ├── Analysis code only. CRA?            |       |         |        |         |         |     |     |
   |  ├── No...............................|   L6  |    -    |    -   |    -    |    ✔    |  ✔  |  x  |
   |  └── Yes..............................|   L7  |    -    |    -   |    -    |    ✔    |  -  |  x  |
   └── A. and cleaning code. Is it CRR?    |       |         |        |         |         |     |     |
       ├── No. CRA?                        |       |         |        |         |         |     |     |
       |  ├── No...........................|   L8  |    -    |    -   |    -    |    -    |  ✔  |  ✔  |
       |  └── Yes..........................|   L9  |    -    |    -   |    -    |    -    |  -  |  ✔  |
       └── Yes.............................|   L10 |    -    |    -   |    -    |    -    |  -  |  -  |
\end{verbatim}

Choose the appropiate level of computational reproducibility and record it using the following format.

\begin{verbatim}
      |-------------|-------|------------------------|------------|
      | output_name | level | additional_explanation | other_info |
      |-------------|-------|------------------------|------------|
      | table 1     | 4     |                        | ...        |
      |-------------|-------|------------------------|------------|
      | table 2     | 7     |                        | ...        |
      |-------------|-------|------------------------|------------|
      | figure 1    | 5     |                        | ...        |
      |-------------|-------|------------------------|------------|
      | ...         | ...   | ...                    | ...        |
      |-------------|-------|------------------------|------------|
\end{verbatim}

Record all this information in the \href{https://docs.google.com/spreadsheets/d/1LUIdVFH0OfR70C7z07TYeE-uWzKI_JIeWUMaYhqEKK0/edit\#gid=1384504774\&range=A1}{assessment tool}. You will be asked to provide this information in the\href{ADD\%20LINK}{assessment and improvement survey}.

\hypertarget{reproducibility-dimensions-at-the-paper-level}{%
\subsection{Reproducibility dimensions at the paper level}\label{reproducibility-dimensions-at-the-paper-level}}

In addition to an output-specific assessment of computational reproducibility, there are several practices that facilitate the overall computational reproducibility of the paper. These practices are described in detail in the Improvement chapter. In this section of assessment it is only required that you verify that the original reproduction package made use of any of the following:

\begin{itemize}
\tightlist
\item
  version control\\
\item
  dynamic document\\
\item
  translate to open source software\\
\item
  file organization\\
\item
  computing capsule (e.g.~CodeOcean, Binder, etc.)
\end{itemize}

Congratulations! You have now completed the Assessment stage. You just provided a concrete building block of knowledge to better understand the state of reproducibility in Economics.

Please continue to the \protect\hyperlink{improvements}{next section} where you can help to improve it!

\hypertarget{improvements}{%
\chapter{Improvements}\label{improvements}}

After completing the assessment of the current reproducibility packages, it is possible to propose ways to increase the reproducibility of the paper. Creating improvements provides an opportunity to gain a more in-depth knowledge of the paper, including its methods, findings, and its overall contribution. In addition to this individual benefits, each contribution will be assessed by the ACRE community and can potentially be used by students and researchers around the world as an improved version of the reproducibility package for the paper.

As in the Assessment section, we recommend that you first focus on one specific output (e.g. ``Table 1''). After completing the improvements to this first output, you will have a much easier time translating those improvements to other outputs.

\hypertarget{types-of-output-level-improvements}{%
\section{Types of output-level improvements}\label{types-of-output-level-improvements}}

\hypertarget{rd}{%
\subsection{Add missing raw data files or meta-data (+RD)}\label{rd}}

It is common that reproductions packages do not include all the original \protect\hyperlink{describe-inputs}{raw datasets}. To obtain any missing raw data, or information about them, follow these steps:

\begin{itemize}
\tightlist
\item
  1 - Identify a specific missing file. During the \protect\hyperlink{assessment}{assessment stage}, you identified all data sources from the paper and appendices (column \texttt{data\_source} in \href{https://docs.google.com/spreadsheets/d/1LUIdVFH0OfR70C7z07TYeE-uWzKI_JIeWUMaYhqEKK0/edit\#gid=0\&range=A1}{this standarized spreadsheet}). However, some data sources (as delivered to the original investigators) might be missing one or more data files. You can sometimes find the specific name of those files by looking at the beginning of the cleaning code scripts. If you find the name of the file, record it in the \texttt{data\_file} field of the \href{https://docs.google.com/spreadsheets/d/1LUIdVFH0OfR70C7z07TYeE-uWzKI_JIeWUMaYhqEKK0/edit\#gid=0\&range=A1}{same spreadsheet} as above. If not, recorded as to ``Some (or all) of the files used in the paper corresponding to the data source X''.\\
\item
  2 - Verify whether this file(s) can be easily obtained from the web.

  \begin{itemize}
  \tightlist
  \item
    2.1 - If yes: obtain the missing files and add them to the reproducibility package. Make sure to obtain permission to repost this data. See \protect\hyperlink{tips-for-communication}{tips for communication} for a template email.\\
  \item
    2.2 - If no: proceed to step 3.\\
  \end{itemize}
\item
  3 - \href{ADD\%20LINK}{Verify the ACRE database} for previous attempts to contact the authors regarding this paper.\\
\item
  4 - Contact the original authors and kindly request the original materials. Be mindful of the authors' time, and remember that the paper you are trying to reproduce was possibly published at a time where standards for computational reproducibility were different. See \protect\hyperlink{tips-for-communication}{tips for communication} for sample language on how to approach the authors.\\
\item
  5 - If the data sets are not available due to confidentiality or proprietary issues, the researcher conducting the reproduction can still improve the reproduction package by providing detailed instructions, including contact information and possible costs, for future researchers to follow.
\end{itemize}

In additions to the efforts to obtain raw data, you can also contribute by obtainning missing analysis data.

\hypertarget{ad}{%
\subsection{Add missing analysis data files (+AD)}\label{ad}}

\protect\hyperlink{describe-inputs}{Analysis data} can be missing for two reasons: (i) raw data exists, but the procedures to transform it into analysis data are not fully reproducible, or (ii) some or all raw data is missing and some of all the analysis data is not included in the original reproduction package. To obtain any missing analysis data, follow these steps:

\begin{itemize}
\tightlist
\item
  1 - Identify the specific name of the missing data set. Typically this information can be found in some of the analysis code that calls such data in order to perform an analysis (eg \texttt{analysis\_data\_03.csv}).\\
\item
  2 - Verify that such data cannot be obtained by running the data cleaning code over the raw data.\\
\item
  3 - \href{ADD\%20LINK}{Verify the ACRE database} for previous attempts to contact the authors on this topic.\\
\item
  4 - \protect\hyperlink{tips-for-communication}{Contact the authors} and request the specific data set.
\end{itemize}

\hypertarget{ac}{%
\subsection{Add missing analysis code (+AC)}\label{ac}}

\protect\hyperlink{describe-inputs}{Analysis code} can be added when there are analytic data files, but some or all the methodological steps are missing from the code. In this case, follow these steps:

\begin{itemize}
\item
  1 - Identify the specific line/paragraph in the paper that describes the analytic step that is missing from the code (eg ``we impute missing values to\ldots{},'' or ``we estimate this regression using a bandwidth of \ldots{}'').
\item
  2 - Identify the code file and the approximate line in the script where the analysis can be carried out. If no relevant code file is found, identify the location of the missing file relative to the steps in the \protect\hyperlink{diagram}{reproduction diagram}.
\item
  3 - \href{ADD\%20LINK}{Verify the ACRE database} for previous attempts to contact the authors on this issue.
\item
  4 - \protect\hyperlink{tips-for-communication}{Contact the authors} nd request the specific code files.
\item
  5 - If no response from the authors, researchers reproducing the paper are encourage to attempt to recreate the analysis based on their interpretation of the paper, and filling in any missing piece by making explicit assumptions.
\end{itemize}

\hypertarget{cc}{%
\subsection{Add missing data cleaning code (+CC)}\label{cc}}

\protect\hyperlink{describe-inputs}{Data cleaning (processing)} code can be added when there are certain steps missing in the creation/re-coding of variables, merging, subsetting of the data sets, and other steps related to data cleaning and processingResearchers conducting the reproduction should follow the same steps (1-5) as when adding missing analysis code.

\hypertarget{dac}{%
\subsection{Debug analysis code (DAC)}\label{dac}}

Whenever any code is available in the reproduction package, reproducers should be able to debug those scripts. There are four types of debugging that you can add as part of the Improvements step:

\begin{itemize}
\tightlist
\item
  \emph{Code cleaning:} instructions are simplified (e.g.~by wrapping repetitive steps in a function or a loop) or redundant code is removed (eg. old code that was commented out), while keeping the original output intact.\\
\item
  \emph{Performance improvement:} instructions are replaced by other that perform the same tasks but take less time (eg. choosing one numerical optimization algorithm over another, but obtaining the same results).\\
\item
  \emph{Environment set up:} the code is modified to include correct paths to files, specific versions of software, and instructions to install missing packages or libraries.\\
\item
  \emph{Correcting errors:} a coding error will occur when a section in the code of the reproduction package executes a procedure that is in direct contradiction with the intended procedure expressed in the documentation (paper or comments of the code). For example, an error happens if the paper specifies that the analysis is performed on the population of males, but the code restricts the analysis to females only. Please follow the \href{ADD\%20LINK}{ACRE procedure to report coding errors}.
\end{itemize}

\hypertarget{debug-cleaning-code-dcc}{%
\subsection{Debug cleaning code (DCC)}\label{debug-cleaning-code-dcc}}

Same as for analysis code, only separate for reporting purposes.

\hypertarget{reporting-results}{%
\subsection{Reporting results}\label{reporting-results}}

Track all the different types of improvements implemented and record in \href{https://docs.google.com/spreadsheets/d/1LUIdVFH0OfR70C7z07TYeE-uWzKI_JIeWUMaYhqEKK0/edit\#gid=0\&range=A3}{this standarized spreadsheet} of the assessment tool with the following structure:

\begin{verbatim}
       Level-specific quality improvements: add data/code, debug code.

       | output_name | imprv | description_of_added_files        | lvl |
       |-------------|-------|-----------------------------------|-----|
       | table 1     | +AD   |        ADD EXAMPLES               |  5  |
       | table 1     | +RD   |        ADD EXAMPLES               |  5  |
       | table 1     | DCC   |        ADD EXAMPLES               |  5  |
       | figure 1    | +CC   |                                   |  6  |
       | figure 1    | DAC   |                                   |  6  |
       | inline 1    | DAC   |                                   |  8  |
       | ...         | ...   | ...                               | ... |  
\end{verbatim}

\hypertarget{paper-level}{%
\section{Types of paper-level improvements}\label{paper-level}}

In addition to the different levels of computational reproducibility described in the previous sections, there are at least six additional that you can implement to improve the overall reproducibility of a paper. Such additional improvements can be applied across levels (including level 10).

\begin{itemize}
\tightlist
\item
  1 - Set up the replication package using version control software (Git).
\item
  2 - Improve documentation: add extensive comments to the code.
\item
  3 - Integrate documentation with code: adapt the paper into a literate programming environment (eg: using Jupyter notebooks, RMarkdown, Stata Dynamic Doc).
\item
  4 - Re-write the code from proprietary statistical software (eg Stata, Matlab) into an open-source statistical software (eg R, Python, Julia).
\item
  5 - Re re-organize the reproduction materials into a set of folders and sub-folders that follow \href{https://www.projecttier.org/tier-protocol/specifications/\#overview-of-the-documentation}{standardized best practices}, and add a master script that executes all the code in order and with no further modifications. \href{https://github.com/AEADataEditor/replication-template}{See AEA's reproduction template}.\\
\item
  6 - Set up a computing capsule that executes all the reproduction in the browser without the need to install any software. See for examples \href{https://mybinder.org/}{Binder} and \href{https://codeocean.com/}{Code Ocean}.
\end{itemize}

\hypertarget{reporting-improvements}{%
\subsection{Reporting improvements}\label{reporting-improvements}}

Repoductors will be asked to provide this information in the \href{ADD\%20LINK}{assessment and improvement survey}.

\hypertarget{robust}{%
\chapter{Checking for Robustness}\label{robust}}

{[}UNDER CONSTRUCION{]}

\begin{itemize}
\tightlist
\item
  Identify all possible analytical choices: original and repeated ones.\\
\item
  Identify type of choice.\\
\item
  Identify choice value.
\item
  Suggest choice alternative and justify (one line)
\end{itemize}

\hypertarget{id-analy}{%
\section{Identifying Analytical Choices}\label{id-analy}}

As part of the requirements to \href{requirements_comprehension.md}{demonstrate comprehension of the paper and the code} researchers conducting the reproduction will be asked to record all the analytical choices identified during the code review process. This is done in two steps: first adding comment lines into the code files where an analytic choice are found, and second, compiling those analytic choices into a standardized data set.

In your copy of the replication code, add the comment \texttt{“\#\ ANALYTICAL\ CHOICE\ OF\ TYPE\ \_\_\_\_.\ RECORDED\ FOR\ THE\ FIRST\ TIME\ {[}HERE\ or\ IN\ "FILE\_NAME-LINE\_NUMBER"{]}”} above each analytical choice detected in the code. Possible types of analytical choices include (but are not limited to):

\begin{itemize}
\tightlist
\item
  Analytical choices in data cleaning code:

  \begin{itemize}
  \tightlist
  \item
    Variable definition\\
  \item
    Data sub-setting\\
  \item
    Data reshaping (merge, append, long/gather, wide/spread)\\
  \item
    Others (specify as ``processing - other'')
  \end{itemize}
\item
  Analytical choices in analysis code:

  \begin{itemize}
  \tightlist
  \item
    Regression function (link function)\\
  \item
    Key parameters (tuning, tolerance parameters, etc.)\\
  \item
    Controls\\
  \item
    Adjustment of standard errors\\
  \item
    Choice of weights\\
  \item
    Treatment of missing values\\
  \item
    Imputations
  \item
    Other (specify as ``methods - other'')
  \end{itemize}
\end{itemize}

Once finished, transcribe all the information on analytical choices into a data set. For the \texttt{source} field type ``original'' whenever the analytical choice is identified for the first time, and \texttt{file\_name-line\ number} every time that the same analytical choice is applied subsequently (for example if a analytic choice is identified for the first time in line 103 and for a second in line 122 their respective values for the \texttt{source} field should be \texttt{original} and \texttt{code\_01.do-L103} respectively).

The resulting data base should have the \href{https://docs.google.com/spreadsheets/d/1nZuJSHswbZgaaIfBcyIUGPwG-WIP8zE1Oambud-WoDc/edit?usp=sharing}{following structure}:

\begin{longtable}[]{@{}lllll@{}}
\toprule
\begin{minipage}[b]{0.10\columnwidth}\raggedright
file\_name\strut
\end{minipage} & \begin{minipage}[b]{0.11\columnwidth}\raggedright
line\_number\strut
\end{minipage} & \begin{minipage}[b]{0.18\columnwidth}\raggedright
choice\_type\strut
\end{minipage} & \begin{minipage}[b]{0.28\columnwidth}\raggedright
choice\_value\strut
\end{minipage} & \begin{minipage}[b]{0.18\columnwidth}\raggedright
Source\strut
\end{minipage}\tabularnewline
\midrule
\endhead
\begin{minipage}[t]{0.10\columnwidth}\raggedright
code\_01.do\strut
\end{minipage} & \begin{minipage}[t]{0.11\columnwidth}\raggedright
73\strut
\end{minipage} & \begin{minipage}[t]{0.18\columnwidth}\raggedright
data subsetting\strut
\end{minipage} & \begin{minipage}[t]{0.28\columnwidth}\raggedright
males\strut
\end{minipage} & \begin{minipage}[t]{0.18\columnwidth}\raggedright
original\strut
\end{minipage}\tabularnewline
\begin{minipage}[t]{0.10\columnwidth}\raggedright
code\_01.do\strut
\end{minipage} & \begin{minipage}[t]{0.11\columnwidth}\raggedright
122\strut
\end{minipage} & \begin{minipage}[t]{0.18\columnwidth}\raggedright
variable definition\strut
\end{minipage} & \begin{minipage}[t]{0.28\columnwidth}\raggedright
income = wages + capital gains\strut
\end{minipage} & \begin{minipage}[t]{0.18\columnwidth}\raggedright
``code\_01.do-L103''\strut
\end{minipage}\tabularnewline
\begin{minipage}[t]{0.10\columnwidth}\raggedright
code\_05.R\strut
\end{minipage} & \begin{minipage}[t]{0.11\columnwidth}\raggedright
143\strut
\end{minipage} & \begin{minipage}[t]{0.18\columnwidth}\raggedright
controls\strut
\end{minipage} & \begin{minipage}[t]{0.28\columnwidth}\raggedright
age, income, education\strut
\end{minipage} & \begin{minipage}[t]{0.18\columnwidth}\raggedright
original\strut
\end{minipage}\tabularnewline
\begin{minipage}[t]{0.10\columnwidth}\raggedright
\ldots{}\strut
\end{minipage} & \begin{minipage}[t]{0.11\columnwidth}\raggedright
\ldots{}\strut
\end{minipage} & \begin{minipage}[t]{0.18\columnwidth}\raggedright
\ldots{}\strut
\end{minipage} & \begin{minipage}[t]{0.28\columnwidth}\raggedright
\ldots{}\strut
\end{minipage} & \begin{minipage}[t]{0.18\columnwidth}\raggedright
\ldots{}\strut
\end{minipage}\tabularnewline
\bottomrule
\end{longtable}

\hypertarget{id-analy}{%
\section{Identifying Choice Type}\label{id-analy}}

\hypertarget{id-analy}{%
\section{Identifying Analytical Choices}\label{id-analy}}

\hypertarget{id-val}{%
\section{Identifying Analytical Value}\label{id-val}}

\hypertarget{test-rob}{%
\section{Choose and justify alternative values for analytical choices}\label{test-rob}}

\hypertarget{test-the-robustness-of-results}{%
\section{Test the robustness of results}\label{test-the-robustness-of-results}}

Test the robustness of results to alternative (sensible) specifications

\begin{itemize}
\tightlist
\item
  Identify sensible alternatives to analytical choices.
\item
  Sample from sensible analytical choices and re-run: report how much do results change as fraction of standard deviations.\\
\item
  Jackknife the preferred estimate.
\item
  Use ML to select among covariates\ldots{}
\end{itemize}

\hypertarget{concluding-the-reproduction}{%
\chapter{Concluding the reproduction}\label{concluding-the-reproduction}}

{[}UNDER CONSTRUCTION{]}

Walk the students on checking that they have completed all the steps and where can they see their output.

\hypertarget{final-products}{%
\section{Final products}\label{final-products}}

\begin{itemize}
\item
  One-page introduction describing why you chose this paper
\item
  Two-page summary of paper
\item
  2 Completed surveys:

  \begin{itemize}
  \tightlist
  \item
    i - General information about the paper and specific
    information about output to reproduce.\\
  \item
    ii - Assessment of how (computationally) reproducible is the paper;
    description of improvements to its reproducibility; record of all the
    analytical choices identified in the exercise.
  \end{itemize}
\item
  ACRE report card with all the improvements that were created by the researcher reproducing the paper. The list of improvements will be made public and original authors will receive a copy of the report card. The option of anonymity will be provided to the researchers reproducing the paper.
\item
  New Readme file (autogenerated).
\item
  Data with all analytical choices identified.
\item
  ?? Narrated description of improvements to original CR of the paper, assessment of robustness of results. Lessons from the exercise and possible future extensions.
\end{itemize}

\hypertarget{guidance-for-a-constructive-exchange-between-reproducers-and-original-authors}{%
\chapter{Guidance for a Constructive Exchange Between Reproducers and Original Authors}\label{guidance-for-a-constructive-exchange-between-reproducers-and-original-authors}}

The purpose of this chapter is to facilitate constructive and respectful communication between reproducers and the original authors. Exchanges that contain charged or/and adversarial language can damage professional relationships and hamper scientific progress. Janz and Freese (\href{https://www.mzes.uni-mannheim.de/openscience/wp-content/uploads/2019/01/Janz-Freese_-Good-and-Bad-Replications-1.pdf}{2019}) articulate two important steps that reproducers can take to ensure that their interactions with original authors are constructive. We provide a summary below and encourage you to follow this guidance. Remember the golden rule of reproductions (and replications): treat others and their work, as you would like others to treat you and your work!

\textbf{1. Carefully and transparently plan your study. }

\begin{enumerate}
\def\labelenumi{\alph{enumi}.}
\tightlist
\item
  Clearly state that you are conducting a reproduction of the original work.\\
\item
  Explain why you have chosen this study. Try to proactively address potential concerns for selection bias.\\
\item
  Explain how ``far'' your results must deviate from the original work before claiming that the study could not be reproduced. Engage deeply with the substantive literature to ensure that your interpretation of differences between the original and reproduction is thorough and acceptable to other authors in the field.
\end{enumerate}

\textbf{2. Use professional and sensitive language. Discuss potential discrepancies between your work and the original paper just like you would have done for your own work.}

\begin{enumerate}
\def\labelenumi{\alph{enumi}.}
\tightlist
\item
  Avoid binary judgments such as ``failed to reproduce,'' and clearly state which results reproduced, and which did not (e.g. ``we successfully reproduced X, but failed to reproduce Y''), unless you uncover a case of apparent scientific misconduct (e.g.~see \href{https://osf.io/qy2se/}{Broockman, Kalla and Aronow, 2015}).\\
\item
  Talk about \emph{the study, not the author}, to avoid making it personal. Make clear what the positive contribution of the original article is. Consider sending a copy of your reproduction report to the original authors.\\
\item
  Discuss what the literature learns from your reproduction, and refrain from claiming to give the final answer to the question.\\
\item
  For papers published five or more years ago, be mindful that norms for reproducibility have evolved since then.\\
\item
  Remember, \emph{the goal is not to criticize previous work or hunt for errors, but to move the literature forward!}
\end{enumerate}

To help put these recommendations into practice, we provide template language for common scenarios that reproducers and authors may encounter in their interactions.

\textbf{For reproducers:}

A. Contacting the original author(s) to request items that are missing;\\
B. Asking for additional guidance when some materials have been shared;\\
C. Response when the original author has refused to share an item due to undisclosed reasons;\\
D. Response when the original author has refused to share data due to legal or ethical constraints;\\
E. Contacting the original author to share the results of your reproduction exercise;\\
F. Responding to adversarial responses from original authors.

\textbf{For authors:}

G. Responding to a request {[}TODO{]}\\
H. Acknowledging that some information is missing {[}TODO{]}\\
I. Acknowledging that some material is still embargoed for future research {[}TODO{]}\\
J. Responding to incomplete/aggressive requests from reproducers {[}TODO{]}

We hope that you will find these useful. At the same time, note that they are \emph{only recommendations}, and you are welcome to modify them based on the context and the needs of your project. Feel free to contact us if you need more guidance or would like to provide feedback on these materials.

\hypertarget{for-reproducers-contacting-the-authors-of-the-original-study}{%
\section{For Reproducers Contacting the Authors of the Original Study}\label{for-reproducers-contacting-the-authors-of-the-original-study}}

\hypertarget{before-you-contact-the-original-author}{%
\subsection{Before you contact the original author}\label{before-you-contact-the-original-author}}

Considering the following:

\begin{enumerate}
\def\labelenumi{\arabic{enumi}.}
\item
  Carefully read all footnotes, appendices, tables, captions, etc. to learn if, how, and where reproduction materials are provided. Follow this \href{https://social-science-data-editors.github.io/guidance/Verification_guidance.html}{Data and Code Guidance} to determine whether you have everything before you start. A few things to consider:

  \begin{itemize}
  \tightlist
  \item
    A Readme file, if available, would be a good place to start. All papers published in AEA journals after July 2019 should have a Readme file.\\
  \item
    Check whether there are any restrictions to accessing the data or code, and whether there are instructions on how to access these files for the purpose of reproduction.
  \end{itemize}
\item
  If reproduction materials are not readily available in the location where the article is published (e.g.~the journal website), check the author'(s) website, Dataverse profile, the ICPSR Publications Related Archive, and other relevant archives and/or data repositories.
\item
  If steps 1 and 2 don't work, then contact the corresponding author (copying the co-authors, if any), and try to consolidate your requests into as few emails as possible. In your email, make sure to include the following details:

  \begin{itemize}
  \tightlist
  \item
    Basic information about the paper being reproduced (include version, date, and a DOI link (or just a URL));\\
  \item
    Context for the reproduction (as part of a class exercise, thesis, etc.) and a notice that the outcome will be recorded in the ACRE reproducibility database;\\
  \item
    Items from the reproduction package that are missing, as well as locations where you had (unsuccessfully) searched for them;\\
  \item
    Use plan: Will the materials be used exclusively for this project? Ask for permission to share the data publicly.\\
  \item
    Right to consultation and results: Will you share the outcome of the reproduction exercise with the original authors?\\
  \item
    A deadline to respond (we suggest at two weeks or longer).
  \end{itemize}
\item
  Follow up if you don't get a response within two weeks, and include any details/clarifications that were left out in your first email.
\end{enumerate}

\hypertarget{a.-contacting-the-original-authors-to-request-items-that-are-missing}{%
\subsection{A. Contacting the original author(s) to request items that are missing}\label{a.-contacting-the-original-authors-to-request-items-that-are-missing}}

\textbf{Template email:}

\begin{quote}
\textbf{Subject:} Reproduction materials for \texttt{{[}“Title\ of\ the\ paper”{]}}
\end{quote}

\begin{quote}
Dear Dr. \texttt{{[}Lastname\ of\ Corresponding\ Author{]}},

I am contacting you about reproduction materials for your paper titled \texttt{{[}Title{]}} which was published in \texttt{{[}Journal{]}} in \texttt{{[}year{]}} (vol \texttt{{[}volume{]}}, no. \texttt{{[}no.{]}}), \texttt{{[}link{]}}.
I am a \texttt{{[}graduate\ student/postdoc/other\ position{]}} at \texttt{{[}Institution{]}}, and I'm working to reproduce this paper as part of a class exercise. \texttt{{[}Add\ context\ for\ why\ you\ want\ to\ reproduce\ this\ particular\ paper\ using\ neutral\ language\ (e.g.\ this\ is\ a\ seminal\ paper\ in\ my\ field),\ avoiding\ any\ statements\ that\ would\ put\ the\ respondent\ on\ the\ defensive{]}}.

To be able to reproduce the paper in full, I hope that you can share the following items: {[}\texttt{list\ items\ missing\ from\ reproduction\ package,\ preferably\ bulleted\ if\ more\ than\ one\ (raw/analytic\ data,\ code,\ protocols\ for\ conducting\ the\ experiment,\ etc.)}{]}. I have already searched \texttt{{[}locations\ where\ you\ searched\ for\ items,\ with\ links\ provided{]}}, however I was not able to locate those items. You can be assured that I will not share any of the materials without your permission, and I will use them exclusively for the purpose of this exercise. Let me know if there are any legal or ethical restrictions that apply to all or parts of the reproduction materials so that I can take that into consideration during this exercise.

Note that I will record the outcome of my reproduction on the Advancing Computational Reproducibility in Economics (ACRE) platform {[}ADD LINK{]}, an online catalog of reproduction projects in economics. ACRE is hosted by the \href{https://www.bitss.org/}{Berkeley Initiative for Transparency in the Social Sciences (BITSS)}. Let me know if you would like me to share the outcome of my exercise with you {[}and whether you are interested in providing a response.

Since I am required to complete this project by \texttt{{[}date{]}}, I would appreciate your response no later than \texttt{{[}deadline{]}}.

Let me know if you have any questions. Please also feel free to contact my supervisor/instructor \texttt{{[}Name\ (email){]}} for further details on this exercise. Thank you in advance for your help!

Best regards,\\
\texttt{{[}Reproducer{]}}
\end{quote}

\hypertarget{b.-asking-for-additional-guidance-when-some-materials-have-been-shared}{%
\subsection{B. Asking for additional guidance when some materials have been shared}\label{b.-asking-for-additional-guidance-when-some-materials-have-been-shared}}

\emph{Note:} Even in cases where the corresponding author has shared the reproduction package, you may still run into challenges in interpreting or executing the materials. That shouldn't discourage you from asking the corresponding author to provide clarifications and/or share missing materials. Just like in the first scenario above, demonstrate that you made an honest effort to reproduce the work using the available resources and try to consolidate your requests into as few emails as possible.

\textbf{Template email:}

\begin{quote}
\textbf{Subject:} Clarification for reproduction materials for \texttt{{[}“Title\ of\ the\ paper”{]}}
\end{quote}

\begin{quote}
Dear Dr. \texttt{{[}Lastname\ of\ Corresponding\ Author{]}},

Thank you for sharing the materials. They have been immensely helpful for my work.

Unfortunately, I ran into a few issues as I delved into the reproduction exercise, and I think your guidance would be helpful in resolving them. \texttt{{[}Describe\ the\ issues\ and\ how\ you\ have\ tried\ to\ resolve\ them.\ Describe\ whatever\ files\ or\ parts\ of\ the\ data\ and/or\ code\ are\ missing.\ Refer\ to\ examples\ B.1\ and\ B.2\ below\ for\ more\ details{]}}.

Thank you in advance for your help.

Best regards,\\
\texttt{{[}Reproducer{]}}
\end{quote}

\hypertarget{b.1-an-example-of-well-described-issues}{%
\subsubsection{B.1 An example of well described issues:}\label{b.1-an-example-of-well-described-issues}}

\begin{quote}
Specifically, I am attempting to reproduce OUTPUT X (eg table 1, figure 3). I found that the following components are required to reproduce to reproduce OUTPUT X:
\end{quote}

\begin{verbatim}
     OUTPUT X
        └───[code] formatting_table1.R
            ├───output1_part1.txt  
            |   └───[code] output_table1.do           
            |       └───[data] analysis_data01.csv
            |          └───[code] data_cleaning01.R*
            |             └───[data] UNKNOWN
            └───output1_part2.txt  
                └───[code] output_table2.do           
                    └───[data] analysis_data02.csv
                       └───[code] data_cleaning02.R
                          └───[data] admin_01raw.csv* 
\end{verbatim}

\begin{quote}
I have marked with an asterix (`` * '') the items that I could not find in the reproduction materials: \textbf{data\_cleaning01.R} and \textbf{admin\_01raw.csv}. After accessing these files, I will also be able to identify the name of the raw data set required to obtain output1\_part1.txt. This is to let you know that and may have to contact you again in case I cannot find this file (labeled as UNKNOWN above) in the reproduction materials.

I understand that this request will require some work for you or somebody in your research group, but I want to assure you that I will add these missing files to the reproducibility package for your paper on the ACRE platform. \textbf{Doing this will ensure that you will not be asked twice for the same missing file.}
\end{quote}

\hypertarget{b.2-an-example-of-poorly-described-issues}{%
\subsubsection{B.2 An example of poorly described issues:}\label{b.2-an-example-of-poorly-described-issues}}

\begin{quote}
Your paper does not reproduce. I have tried for several hours now, and can't get the DO files to run. Could you please share all the missing reproduction materials? Data and code sharing are a basic principle of open science, so I am confident that you will do the right thing.
\end{quote}

\hypertarget{c.-response-when-the-original-author-has-refused-to-share-due-to-undisclosed-reasons}{%
\subsection{C. Response when the original author has refused to share due to undisclosed reasons}\label{c.-response-when-the-original-author-has-refused-to-share-due-to-undisclosed-reasons}}

\emph{Note:} You can also use this template in cases where the corresponding author has not submitted a response after two or more follow-up emails.

\textbf{Template email:}

\begin{quote}
\textbf{Subject:} Re: Reproduction materials for `{[}``Title of the paper''{]}
\end{quote}

\begin{quote}
Dear Dr. \texttt{{[}Lastname\ of\ Corresponding\ Author{]}},

Thank you for your response. I will try to reproduce the paper using the available materials, and will record the missing items accordingly on the ACRE platform. l will also post my assessment of the reproducibility of the paper in its current form based on the ACRE reproducibility scale{[}ADD LINK{]}.

Let me know if you have any questions.

Best regards,\\
\textless{}\textgreater{}
\end{quote}

\hypertarget{d.-response-when-the-original-author-has-refused-to-share-due-to-legal-or-ethical-restrictions-of-the-data}{%
\subsection{D. Response when the original author has refused to share due to legal or ethical restrictions of the data}\label{d.-response-when-the-original-author-has-refused-to-share-due-to-legal-or-ethical-restrictions-of-the-data}}

\textbf{Template email:}

\begin{quote}
\textbf{Subject:} Re: Reproduction materials for `{[}``Title of the paper''{]}
\end{quote}

\begin{quote}
Dear Dr. \texttt{{[}Lastname\ of\ Corresponding\ Author{]}},

Thank you for your response, and for clarifying the terms of use of the reproduction materials.

Though I understand that you are unable to share the raw data, there may be alternative steps you can take that would help me improve the reproducibility of your paper. These include:

\begin{enumerate}
\def\labelenumi{\arabic{enumi}.}
\tightlist
\item
  Sharing the analytic version of the data (the version of the dataset that was used in the final version of your paper);\\
\item
  Providing a public description of the steps other researchers can follow to request access to the raw data or materials, including an estimate of the cost and the duration of the process (find an example/checklist here{[}ADD EXAMPLE{]}); and\\
\item
  Providing access to all data and materials for which the constraints do not apply.
\end{enumerate}

Based on my assessment, your paper would currently rank at \texttt{{[}level\ X{]}} on the ACRE reproducibility scale{[}ADD LINK{]}, however \emph{this score can be easily improved}. Being able to provide analytic data would elevate the reproducibility of your paper to \texttt{{[}level\ Y{]}}. Providing public instructions on how other parties can access the data would further elevate its reproducibility to \texttt{{[}level\ Z{]}}.

I would be happy to help if you are interested in taking any of the steps I outlined above. Let me know if that would be helpful.

Thank you for your help!

Best regards,\\
\texttt{{[}Reproducer{]}}
\end{quote}

\hypertarget{e.-contacting-the-original-author-to-share-the-results-of-your-reproduction-exercise}{%
\subsection{E. Contacting the original author to share the results of your reproduction exercise}\label{e.-contacting-the-original-author-to-share-the-results-of-your-reproduction-exercise}}

{[}WORK IN PROGRESS{]}

\hypertarget{f.-responding-to-hostile-responses-from-original-authors}{%
\subsection{F. Responding to hostile responses from original authors}\label{f.-responding-to-hostile-responses-from-original-authors}}

\emph{Note:} Planning your study carefully and transparently, and using professional and sensitive language are the best ways to ensure that the interaction will be beneficial to both you and the original author. However, unpleasant interactions may happen despite your best efforts, and can range anywhere from dismissive comments to bullying, discrimination, and harassment.

\hypertarget{dismissive-comments}{%
\subsubsection{Dismissive comments}\label{dismissive-comments}}

In cases of dismissive comments, the best course of action may be to simply thank the author for their response and continue with the exercise.

\textbf{Template email:}

\begin{quote}
\textbf{Subject:} Re: Reproduction materials for `{[}``Title of the paper''{]}
\end{quote}

\begin{quote}
Dear Dr. \texttt{{[}Lastname\ of\ Corresponding\ Author{]}},

Thank you for your response. I will work to reproduce using the available materials, and will record my results accordingly on the ACRE platform. l will also post my assessment of the reproducibility of the paper in its current form based on the ACRE reproducibility scale{[}ADD LINK{]}.

Let me know if you have any questions.

Best regards,\\
\texttt{{[}Reproducer{]}}
\end{quote}

\hypertarget{harassment-andor-discrimination}{%
\subsubsection{Harassment and/or discrimination}\label{harassment-andor-discrimination}}

The AEA and other economic societies have strict policies against harassment and discrimination. Here are some of the behaviors that the \href{https://www.aeaweb.org/about-aea/aea-policy-harassment-discrimination}{AEA Policy on Harassment and Discrimination} has listed as unacceptable, and could emerge in a hostile exchange regarding a reproduction:

\begin{itemize}
\tightlist
\item
  Intentionally intimidating, threatening, harassing, or abusive actions or remarks (both spoken and in other media)
\item
  Prejudicial actions or comments that undermine the principles of equal opportunity, fair treatment, or free academic exchange
\item
  Deliberate intimidation, stalking, or following
\item
  Real or implied threat of physical harm.
\end{itemize}

Here are a some steps you can take if you believe you have bullying, discrimination or harassment:

\begin{itemize}
\tightlist
\item
  \textbf{File a complaint with the \href{https://www.aeaweb.org/about-aea/aea-ombudsperson}{AEA Ombudsperson}.} Any AEA member can file a complaint (you can also join the AEA solely for the purpose of filing a report). The person you are making the complaint about need not be an AEA member. A non-AEA member can also file a report if the act of harassment or discrimination was committed by an AEA member or in the context of an AEA-sponsored activity. Learn more about the process \href{https://www.aeaweb.org/about-aea/aea-ombudsperson/faq}{here}.
\item
  \textbf{File a report with your institution's office for the prevention of harassment \& discrimination.} US-based institutions have internal mechanisms that allow students and faculty to seek support in cases of discrimination and harassment on the basis of categories including race, color, national origin, gender, age, sexual orientation/identity, including allegations of sexual harassment and sexual violence. Formal titles of this office vary across institutions, but common names include ``Office for the Prevention of Harassment and Discrimination'' (in institutions that are part of the University of California) system, ``Office of Equity and Title IX'', etc.
\item
  \textbf{Contact your institution's Ombudsperson/Ombuds Office.} If you believe that you have experienced academic bullying or other forms of disrespectful behavior that fall may outside the scope of harassment and/or discrimination as described above, you should know that university ombuds officers are a confidential, impartial resource to discuss your concerns and learn about potential next steps available in your case.
\item
  \textbf{Access mental health services at your institution.} Many universities offer short-term Counseling \& Psychological Services (CAPS) for academic, career, and personal issues.
\item
  \textbf{Ask for support from your academic supervisor.} If you are unsure on how to proceed, consult your academic supervisor on whether continuing the exercise would be appropriate.
\end{itemize}

\href{https://docs.google.com/document/d/1xJ7pZTQ1VQXVCrs6IUlp7HlBB4oxYha0oOniCG2SWLM/edit?ts=5d251563}{draft here}

\hypertarget{for-original-authors-responding-to-requests-from-reproducers}{%
\section{For Original Authors Responding to Requests from Reproducers}\label{for-original-authors-responding-to-requests-from-reproducers}}

\hypertarget{responding-to-a-repeated-request}{%
\subsection{Responding to a repeated request}\label{responding-to-a-repeated-request}}

{[}TO DO{]}
\#\#\# Acknowledging that some information is missing

{[}TO DO{]}

\hypertarget{acknowledging-that-some-material-is-still-embargoed-for-future-research}{%
\subsection{Acknowledging that some material is still embargoed for future research}\label{acknowledging-that-some-material-is-still-embargoed-for-future-research}}

{[}TO DO{]}

\hypertarget{responding-to-incompleteaggressive-requests-from-reproducer}{%
\subsection{Responding to incomplete/aggressive requests from reproducer}\label{responding-to-incompleteaggressive-requests-from-reproducer}}

\hypertarget{code-of-conduct}{%
\chapter{Code of Conduct}\label{code-of-conduct}}

{[}UNDER CONSTRUCTION{]}

\hypertarget{contributions}{%
\chapter{Contributions}\label{contributions}}

TO DO

\hypertarget{ask-for-feedback-on-guidelines}{%
\section{Ask for feedback on guidelines}\label{ask-for-feedback-on-guidelines}}

\hypertarget{list-of-contributors}{%
\section{List of Contributors}\label{list-of-contributors}}

\hypertarget{reproduction-diagrams}{%
\chapter{Reproduction Diagrams}\label{reproduction-diagrams}}

\hypertarget{different-scenarios}{%
\section{Different Scenarios}\label{different-scenarios}}

JOEL: PLease fill-in.

\hypertarget{complete}{%
\subsection{Complete}\label{complete}}

\begin{verbatim}
      table 1
        └───[code] formatting_table1.R
            ├───output1_part1.txt  
            |   └───[code] output_table1.do           
            |       └───[data] analysis_data01.csv
            |          └───[code] data_cleaning01.R
            |             └───[data] survey_01raw.csv
            └───output1_part2.txt  
                └───[code] output_table2.do           
                    └───[data] analysis_data02.csv
                       └───[code] data_cleaning02.R
                          └───[data] admin_01raw.csv  
\end{verbatim}

\hypertarget{raw-data-and-analytic-data-but-cleaning-code-is-missing.}{%
\subsection{Raw data and analytic data, but cleaning code is missing.}\label{raw-data-and-analytic-data-but-cleaning-code-is-missing.}}

\begin{verbatim}
      table 1
        └───[code] formatting_table1.R
            ├───output1_part1.txt  
            |   └───[code] output_table1.do           
            |       └───[data] analysis_data01.csv
            |          └───[code] MISSING FILE(S)
            |             └───[data] survey_01raw.csv
            └───output1_part2.txt  
                └───[code] output_table2.do           
                    └───[data] analysis_data02.csv
                       └───[code] MISSIN FILE(S)
                          └───[data] admin_01raw.csv  
\end{verbatim}

\hypertarget{additional-resources}{%
\section{Additional resources}\label{additional-resources}}

Create a section with short summaries of great resources for comp. repro and invite reader to contribute.

\hypertarget{some-summaries}{%
\subsection{Some summaries}\label{some-summaries}}

\hypertarget{summary-on-reproducbile-workflow-chapter-11-from-christensen2019transparent}{%
\subsection{\texorpdfstring{Summary on reproducbile workflow (Chapter 11) from \citet{christensen2019transparent}:}{Summary on reproducbile workflow (Chapter 11) from @christensen2019transparent:}}\label{summary-on-reproducbile-workflow-chapter-11-from-christensen2019transparent}}

\begin{itemize}
\tightlist
\item
  TODO
\end{itemize}

\hypertarget{links}{%
\subsection{Links}\label{links}}

\begin{itemize}
\tightlist
\item
  \href{https://www.projecttier.org/tier-protocol/}{Project TIER}\\
\item
  \href{http://idbdocs.iadb.org/wsdocs/getdocument.aspx?docnum=EZSHARE-1350314980-383}{IDB's cheatsheet for transparency, reproducibility and ethics}\\
\item
  \href{https://github.com/labordynamicsinstitute/replicability-training/wiki}{Lars Vilhuber LDI's Wiki for Reproducibility}. Particularly \href{https://github.com/labordynamicsinstitute/replicability-training/wiki/Prepare_and_run_replication}{this section}.\\
\item
  World Bank \href{https://dimewiki.worldbank.org/wiki/Main_Page}{DIME's Wiki for transparent and reproducible research}.
\item
  Dynamic documents in \href{https://rmarkdown.rstudio.com/gallery.html}{R}, \href{https://github.com/jupyter/jupyter/wiki/A-gallery-of-interesting-Jupyter-Notebooks\#economics-and-finance}{Python} and \href{https://github.com/BITSS/CEGA2019/blob/master/03-extra_dynamic_docs/02b-Stata-markdown/Stata\%20Markdown.pdf}{Stata}\\
\item
  Git resources:

  \begin{itemize}
  \tightlist
  \item
    \href{https://happygitwithr.com}{Jenny Bryan's book} and \href{https://www.rstudio.com/resources/videos/happy-git-and-gihub-for-the-user-tutorial/}{video}\\
  \item
    \href{https://lab.github.com/}{Github learning lab}
  \item
    \href{https://www.udacity.com/course/how-to-use-git-and-github--ud775}{Udacity's intro}\\
  \item
    \href{https://www.youtube.com/playlist?list=PLRqwX-V7Uu6ZF9C0YMKuns9sLDzK6zoiV}{Git for poets}\\
  \item
    \href{https://github.com/kbjarkefur/GitHubDropBox}{Combining GitHub and Dropbox}\\
  \item
    \href{https://www.atlassian.com/git/tutorials}{Atlassian intro to Git}
  \item
    \href{https://swcarpentry.github.io/git-novice/}{Software Carpentry tutorial from the command line}
  \end{itemize}
\item
  \href{https://osf.io}{Open Science Framework (OSF)}
\item
  \href{https://github.com/hblackburn/R4Econ/blob/master/Resources.md}{R for Stata users}
\end{itemize}

\hypertarget{additional-resources-1}{%
\chapter{Additional resources}\label{additional-resources-1}}

\begin{itemize}
\tightlist
\item
  \textbf{Coding errors:} a coding error will occur when a section in the code, of the reproduction package, executes a procedure that is in direct contradiction with the intended procedure expressed in the documentation (paper or comments of the code). For example an error happens if the paper specify that the analysis is perform on the population of males, but the code restricts the analysis to females only. Please follow the \href{ADD\%20LINK}{ACRE procedure to report coding errors}.
\end{itemize}

Create a section with short summaries of great resources for comp. repro and invite reader to contribute.

\hypertarget{some-summaries-1}{%
\section{Some summaries}\label{some-summaries-1}}

\hypertarget{summary-on-reproducbile-workflow-chapter-11-from-christensen2019transparent-1}{%
\subsection{\texorpdfstring{Summary on reproducbile workflow (Chapter 11) from \citet{christensen2019transparent}:}{Summary on reproducbile workflow (Chapter 11) from @christensen2019transparent:}}\label{summary-on-reproducbile-workflow-chapter-11-from-christensen2019transparent-1}}

\begin{itemize}
\tightlist
\item
  TODO
\end{itemize}

\hypertarget{links-1}{%
\section{Links}\label{links-1}}

TODO: Add and classify

\begin{itemize}
\tightlist
\item
  \href{https://www.projecttier.org/tier-protocol/}{Project TIER}\\
\item
  \href{http://idbdocs.iadb.org/wsdocs/getdocument.aspx?docnum=EZSHARE-1350314980-383}{IDB's cheatsheet for transparency, reproducibility and ethics}\\
\item
  \href{https://github.com/labordynamicsinstitute/replicability-training/wiki}{Lars Vilhuber LDI's Wiki for Reproducibility}. Particularly \href{https://github.com/labordynamicsinstitute/replicability-training/wiki/Prepare_and_run_replication}{this section}.\\
\item
  World Bank \href{https://dimewiki.worldbank.org/wiki/Main_Page}{DIME's Wiki for transparent and reproducible research}.
\item
  Dynamic documents in \href{https://rmarkdown.rstudio.com/gallery.html}{R}, \href{https://github.com/jupyter/jupyter/wiki/A-gallery-of-interesting-Jupyter-Notebooks\#economics-and-finance}{Python} and \href{https://github.com/BITSS/CEGA2019/blob/master/03-extra_dynamic_docs/02b-Stata-markdown/Stata\%20Markdown.pdf}{Stata}\\
\item
  Git resources:

  \begin{itemize}
  \tightlist
  \item
    \href{https://happygitwithr.com}{Jenny Bryan's book} and \href{https://www.rstudio.com/resources/videos/happy-git-and-gihub-for-the-user-tutorial/}{video}\\
  \item
    \href{https://lab.github.com/}{Github learning lab}
  \item
    \href{https://www.udacity.com/course/how-to-use-git-and-github--ud775}{Udacity's intro}\\
  \item
    \href{https://www.youtube.com/playlist?list=PLRqwX-V7Uu6ZF9C0YMKuns9sLDzK6zoiV}{Git for poets}\\
  \item
    \href{https://github.com/kbjarkefur/GitHubDropBox}{Combining GitHub and Dropbox}\\
  \item
    \href{https://www.atlassian.com/git/tutorials}{Atlassian intro to Git}
  \item
    \href{https://swcarpentry.github.io/git-novice/}{Software Carpentry tutorial from the command line}
  \end{itemize}
\item
  \href{https://osf.io}{Open Science Framework (OSF)}
\item
  \href{https://github.com/hblackburn/R4Econ/blob/master/Resources.md}{R for Stata users}
\end{itemize}

\bibliography{book.bib,packages.bib}

\end{document}
